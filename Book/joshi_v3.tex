% set to 'oneside' for web style, 'twoside' for book print
% for amazon:
% \documentclass[paper=6in:9in,pagesize=pdftex,headinclude=on,footinclude=on,12pt,twoside]{scrbook}
% \areaset[0.50in]{4.5in}{8in}
% for normal size:
%!TEX TS-program = xelatex


\documentclass[a4paper,headinclude=on,footinclude=on,12pt,oneside]{scrbook}

\usepackage[utf8]{inputenc}
\usepackage[T1]{fontenc}
\usepackage{xspace}
\usepackage{bera}
\usepackage{pifont}
\usepackage{amssymb}
\usepackage[dvipsnames]{xcolor}
\usepackage{graphicx}
\graphicspath{ {./images/} }
\usepackage{pgf}
\usepackage{tikz}
\usetikzlibrary{shapes}
\usepackage{color}
\usepackage{textcomp}
\usepackage{float}
\usepackage[driverfallback=dvipdfm]{hyperref}


% for multiple appendices
\usepackage[toc,page]{appendix}

% for index
\usepackage{imakeidx}

% for fancy icons in listing
\usepackage{fontawesome}

% for advanced code highlighting
\usepackage{minted}

% for sizing emoji png's to font height
\usepackage{scalerel}

\usepackage{xparse}

\usepackage{fontspec}





% improve spacing for section listing in table of contents
\makeatletter
\renewcommand*\l@section{\@dottedtocline{1}{1.5em}{3em}}
\makeatother

% for spacing for chapters
\usepackage{tocloft}
\makeatletter
\renewcommand{\numberline}[1]{%
  \@cftbsnum #1\@cftasnum~\@cftasnumb%
}
\makeatother

% for highlighted sections of texts i.e. "frames", as well as code snippets
\usepackage[most,minted]{tcolorbox}
\tcbuselibrary{listings,minted,breakable}

\lstset{
  breaklines=true,
  breakatwhitespace=false,
  xleftmargin=1em,
  frame=single,
  numbers=left,
  numbersep=5pt,
}

% \usepackage{listings}
% % TODO: this doesn't solve the multiple page code snippet issue
% % see maybe https://tex.stackexchange.com/questions/117836/code-listing-spanning-multiple-pages-with-captions-at-top
% \lstset{float=H}

\usetikzlibrary{calc,shadows.blur}

%%%%%%%%%%%%

\DeclareFixedFont{\numcap}{T1}{phv}{bx}{n}{3cm}
\DeclareFixedFont{\capitalizedtext}{T1}{phv}{bx}{n}{1.5cm}
\DeclareFixedFont{\textaut}{T1}{phv}{bx}{n}{0.8cm} 

\addtokomafont{chapter}{\color{gray}\capitalizedtext}
\addtokomafont{section}{\color{white}\small}
\addtokomafont{subsection}{\color{white}\small}
\setkomafont{pagehead}{\sffamily\small}
\setkomafont{captionlabel}{\sffamily\small\bfseries}
\setkomafont{caption}{\sffamily\small}
%%%%%%%%%%%%%%%%%%%%%%%%%%%%%%%%%%%%%%%%%%%%%%
\usetikzlibrary{calc,trees,positioning,arrows,chains,shapes.geometric,
    decorations.pathreplacing,decorations.pathmorphing,shapes,
    matrix,shapes.symbols}

\tikzset{
  punktchain/.style={
    rectangle, 
    rounded corners, 
    draw=black!20, thin,
    minimum height=3em, 
    text centered},
  peu/.style={
    rectangle,
    fill opacity=1,
    %rounded corners, 
    fill=white,
    top color=white,
    draw=black!20, thin,
    %text width=10em, 
    %minimum height=3em, 
    text centered},
  line/.style={draw, thin, <-},
  element/.style={
    tape,
    top color=white,
    bottom color=blue!50!black!60!,
    minimum width=8em,
    draw=blue!40!black!90, very thick,
    text width=10em, 
    minimum height=3.5em, 
    text centered, 
    on chain},
}
%%%%%%%%%%%%%%%%%%%%%%%%%%%%%%%%%%%%%%%%%%%%%%
\usepackage{scrlayer-scrpage}
\setlength{\headheight}{25pt}
\pagestyle{scrheadings}
\addtokomafont{headsepline}{\color{lightgray}}

\lefoot{\color{black!40}{\hrulefill}}
\cefoot{\parbox[c][.5in][c]{1cm}{\fcolorbox{black!40}{white}{\thepage}}}
\refoot{}

\lofoot{\color{black!40}{\hrulefill}}
\cofoot[{\color{black!40}{---}} {\thepage} {\color{black!40}{---}}]{\parbox[c][.5in][c]{1cm}{\fcolorbox{black!40}{white}{\thepage}}}
\rofoot[]{}


\tolerance=4000
\emergencystretch=20pt

\setcounter{secnumdepth}{3}

\usepackage{lipsum}
%%%%%%%%%%%%%%%%%%%%
\usepackage{enumitem}

% for fancy spacing of code snippet title bars
\usepackage{tabularx}
\newcolumntype{\CeX}{>{\centering\let\newline\\\arraybackslash}X}%
\newcommand{\TwoSymbolsAndText}[3]{%
  \begin{tabularx}{\textwidth}{c\CeX c}%
    #1 & #2 & #3
  \end{tabularx}%
}

\newlist{steps}{enumerate}{4}
\setlist[steps]{topsep=0pt,partopsep=0pt,itemsep=0pt,parsep=0pt,labelindent=0.5cm,leftmargin=*}
\setlist[steps,1]{label*=\arabic*.}
\setlist[steps,2]{label*=\arabic*.}
\setlist[steps,3]{label*=\arabic*.}
\setlist[steps,4]{label*=\arabic*.}

\newlist{points}{itemize}{4}
\setlist[points]{topsep=0pt,partopsep=0pt,itemsep=0pt,parsep=0pt,labelindent=0.5cm,leftmargin=*}
\setlist[points,1]{label=\tiny\ding{110}}
\setlist[points,2]{label=\tiny\ding{108}}
\setlist[points,3]{label=\tiny\ding{72}}
\setlist[points,4]{label=\tiny\ding{117}}

\newlist{objectives}{itemize}{1}
\setlist[objectives]{topsep=0pt,partopsep=0pt,itemsep=0pt,parsep=0pt,labelindent=0.5cm,leftmargin=*}
\setlist[objectives,1]{label=\tiny$\blacktriangleright$}

\newlist{attention}{itemize}{1}
\setlist[attention]{topsep=0pt,partopsep=0pt,itemsep=0pt,parsep=0pt,labelindent=0.5cm,leftmargin=*}
\setlist[attention,1]{label=\ding{224}}

\newlist{arrows}{itemize}{4}
\setlist[arrows]{topsep=0pt,partopsep=0pt,itemsep=0pt,parsep=0pt,labelindent=0.5cm,leftmargin=*}
\setlist[arrows,1]{label=\tiny\ding{252}}
\setlist[arrows,2]{label=\tiny\ding{212}}
\setlist[arrows,3]{label=\tiny\ding{232}}
\setlist[arrows,4]{label=\tiny\ding{217}}
%%%%%%%%%%%%%%%%%%%%
\usepackage[tikz]{bclogo}
\renewcommand\logowidth{14pt}

\usepackage{colortbl}
\arrayrulecolor{gray}

% Custom colors
\definecolor{monokaiPink}{HTML}{F92771}
\definecolor{npmred}{HTML}{BB2E3E}
\definecolor{codebackground}{HTML}{F2F2F2}

\usemintedstyle{default}

%%% Custom Commands %%%
\newcommand{\link}[2]{\textbf{\textcolor{monokaiPink}{\href{#2}{#1}}}}

\newcommand{\standardfigure}[3]{\begin{figure}[H]\begin{center}\includegraphics[width=#1]{#2}\caption{#3}\label{fig:#2}\end{center}\end{figure}}

% for that pain in the ass @ symbol
\newcommand{\at}{\makeatletter @\makeatother}

% for the dollar symbol
\newcommand{\dollar}{\$}

% custom command for NPM-like red code snippets
\NewDocumentCommand\codeword{v}{\texttt{\textbf{\textcolor{npmred}{#1}\index{#1}}}}

% juicy code snippets
\AtBeginDocument{
\newtcblisting[blend into=listings]{codeInput}[3]{
  listing engine=minted,
  minted language=#1,
  minted options={breaklines,breaksymbolleft=,breaksymbolright=,fontsize=\footnotesize},
  listing only,
  listing remove caption=true,
  size=title,
  arc=1.5mm,
  breakable,
  enhanced jigsaw,
  colframe=Black,
  coltitle=White,
  boxrule=0.5mm,
  colback=codebackground,
  coltext=Black,
  title=\TwoSymbolsAndText{\faCode}{%
    \footnotesize{\texttt{#2}}
  }{\faCode},
  list text=#3,
  enlarge top initially by=12pt,
  enlarge bottom finally by=8pt
}
\newtcbinputlisting[blend into=listings]{\codeFromFile}[4]{
  listing engine=minted,
  minted language=#1,
  listing file={#4},
  minted options={breaklines,breaksymbolleft=,breaksymbolright=,fontsize=\footnotesize},
  listing only,
  listing remove caption=true,
  size=title,
  arc=1.5mm,
  breakable,
  enhanced jigsaw,
  colframe=Black,
  coltitle=White,
  boxrule=0.5mm,
  colback=codebackground,
  coltext=Black,
  title=\TwoSymbolsAndText{\faCode}{%
    \footnotesize{\texttt{#2}}
  }{\faCode},
  list text=#3,
  enlarge top initially by=12pt,
  enlarge bottom finally by=8pt
}
}

% emoji Commands
\NewDocumentCommand\warning{}{
  \includegraphics[width=0.5cm, height=0.5cm]{images/emojis/u26A0.png}
}

\NewDocumentCommand\information{}{
  \includegraphics[width=0.5cm, height=0.5cm]{images/emojis/u2139.png}
}

\NewDocumentCommand\greenCheck{}{
  \includegraphics[width=0.5cm, height=0.5cm]{images/emojis/u2705.png}
}

\NewDocumentCommand\wink{}{
  \includegraphics[scale=0.05]{emojis/u1F609.png}
}

\NewDocumentCommand\thumbsup{}{
  \includegraphics[scale=0.05]{emojis/u1F44D.0.png}
}

\NewDocumentCommand\rocket{}{
  \includegraphics[scale=0.05]{emojis/u1F680.png}
}

\NewDocumentCommand\beers{}{
  \includegraphics[scale=0.05]{emojis/u1F37B.png}
}

\NewDocumentCommand\joy{}{
  \includegraphics[scale=0.05]{emojis/u1F602.png}
}

\NewDocumentCommand\soup{}{
  \includegraphics[scale=0.05]{emojis/u1F35C.png}
}

\NewDocumentCommand\nuts{}{
  \includegraphics[scale=0.05]{emojis/u1F95C.png}
}

\NewDocumentCommand\partypopper{}{
  \includegraphics[scale=0.05]{emojis/u1F389.png}
}

\newtcolorbox{highlightBox}[4][]{%
  enhanced jigsaw,
  colback=#3!10!white,%
  colframe=black!80!black,
  size=small,
  boxrule=1pt,
  title=\raisebox{-3pt}{#4} \textbf{#2} \raisebox{-3pt}{#4},
  halign title=flush center,
  coltitle=black,
  breakable,
  drop shadow=#3!50!white,
  attach boxed title to top left={xshift=1cm,yshift=-\tcboxedtitleheight/2,yshifttext=-\tcboxedtitleheight/2},
  minipage boxed title=7cm,
  boxed title style={%
    colback=#3!10!white,
    size=fbox,
    boxrule=1pt,
    boxsep=2pt,
    underlay={
      \coordinate (dotA) at ($(interior.west) + (-0.5pt,0)$);
      \coordinate (dotB) at ($(interior.east) + (0.5pt,0)$);
    },
  },
  #1,
  enlarge top initially by=12pt,
  enlarge bottom finally by=8pt
}

% for index (and linking to it)
\makeindex[title={Index\label{index}}]

% to get rid of listoflistings warning
% see https://tex.stackexchange.com/questions/51867/koma-warning-about-toc
% should also be loaded last.... sigh
\usepackage{scrhack}

% fix for code highlighting in sass files
% see: https://tex.stackexchange.com/questions/684739/minted-with-tcolorbox-syntax-highlighting-issue-with-sass-code-snippet/684745
\makeatletter
\AddToHook{cmd/minted@addcachefile/after}{%
\@namedef{PYG@tok@err}{\def\PYG@bc##1{##1}}}
\makeatother

%%%%%%%%%%%%%%%%%%%%%%%%%%%%%%%%%%%%%%%%%%%%%%%%%%%%%%%%%%
\begin{document}

%%%%%%%%%%%%%%%%%%%%%% First Page
\title{\capitalizedtext{Gen AI for Financial Risk}\\\small{Learn Agentically powered Gen AI ;  
		Gen AI Agentic Framework for Financial Risk Analytics !}}
\author{
    \textaut{Satyadhar Joshi}\\https://satyadharjoshi.com
}
\date{\today}

\maketitle
%%%%%%%%%%%%%%%%%%%%%%
\tableofcontents

\listoffigures
\addcontentsline{toc}{chapter}{List of Figures}

\listoflistings
\addcontentsline{toc}{chapter}{List of Listings}

%*************************************************************************
\chapter*{Foreword}
\addcontentsline{toc}{chapter}{Foreword}
%*************************************************************************
\dictum[Isaac Newtown, 1675]{If I have seen further it is by standing on the shoulders of Giants.}

\minisec{Here's a Section Title}

Here is some normal book text, and here are some points:

\begin{arrows}
\item Point one
\item Point two
\item Point three
\end{arrows}

\minisec{Highlight Boxes}

You can make use of these highlight boxes:

\begin{highlightBox}{Green Highlight Boxes}{green}{\greenCheck}
I use green highlight boxes for positive or success milestones in a book.
\end{highlightBox}

\begin{highlightBox}{Blue Highlight Boxes}{blue}{\information}
I use blue highlight boxes for important caveats, information, or tips.
\end{highlightBox}

\begin{highlightBox}{Yellow Highlight Boxes}{yellow}{\warning}
I use yellow highlight boxes for any gotchyas, warnings, or things that could go wrong.
\end{highlightBox}

\minisec{Use the Index, Listings, Recipes, and Figures to Your Advantage}

By the power of LaTeX, a variety of helpful references have been built into this book:



The list of listings also includes every code snippet in the entire book with a detailed description. Use it to jump to whatever snippet you'd like to look at.

Likewise, the list of Recipes is a custom listing of reusable style code that shouldn't need to be refactored away from ReduxPlate - these recipes are generic snippets or files that can be reused in any SaaS product.

\minisec{Are You Ready?}

Something something, let's go!

- Jane Doe

\textit{Town, Country, May 2023}

%*************************************************************************
\chapter{Gen AI in Finance in light of Agentic Framework: Trends }
%*************************************************************************
\dictum%
[Marc Andreessen]%author
{It's really rare for people to have a successful start-up in this industry without a breakthrough product. I'll take it a step further. It has to be a radical product. It has to be something where, when people look at it, at first they say, 'I don't get it, I don't understand it. I think it's too weird, I think it's too unusual.
} %text

\section{Different Finance Areas}



Here's some text in the section

Chapter titles
Intro to LLM in Finance (not just risk but all Finance) 


Data Engineering and Data lakes for Gen AI (need to write)


Agentic Design of Full stack Gen AI system for fin risk analytics,

 Agent-Oriented Architectures for Financial Data Pipelines (need to write) 
 
 
Gen AI in Credit Risk (same paper logistic Regression) Huggin face


Gen AI in Market Risk (gan vae part 1, then calculate VaR)


Gen AI for Interest Rate modeling (gan vae part 2)


Gen AI in Structured Finance (to write- you have the subtopics already - treasury MBS etc models)


Gen AL Prompt Engineering for Financial Risk (use the one you have)


Gen AI in Model Implementation for Financial Risk, (todo )


\section{Implementation Areas}

Here's some text in the section second

%*************************************************************************
\chapter{Financial Risk Management in light of Gen AI aiding Regulation}
%*************************************************************************
\dictum%
[Steve Jobs, 1997]%author
{You've got to start with the customer experience and work backwards to the technology.
} %text

\section{Market and Credit Risk}

In a software book, it's often nice to list chapter objectives at the start of the chapter. I do it this way:

\minisec{Chapter Objectives}
\begin{arrows}
\item Some objective
\item Some other objective
\item Another objective
\end{arrows}

\minisec{Code Snippets}

Code snippets are of course also essential in a dev book. Here is a code snippet:

\begin{codeInput}{bash}{terminal}{Installing Express via npm.}
npm install express
\end{codeInput}

Here's a longer code snippet, an MIT license:

\begin{codeInput}{markdown}{LICENSE}{An example MIT license code snippet.}
MIT License

Copyright (c) Your Company LLC and its affiliates.

Permission is hereby granted, free of charge, to any person obtaining a copy
of this software and associated documentation files (the "Software"), to deal
in the Software without restriction, including without limitation the rights
to use, copy, modify, merge, publish, distribute, sublicense, and/or sell
copies of the Software, and to permit persons to whom the Software is
furnished to do so, subject to the following conditions:

The above copyright notice and this permission notice shall be included in all
copies or substantial portions of the Software.

THE SOFTWARE IS PROVIDED "AS IS", WITHOUT WARRANTY OF ANY KIND, EXPRESS OR
IMPLIED, INCLUDING BUT NOT LIMITED TO THE WARRANTIES OF MERCHANTABILITY,
FITNESS FOR A PARTICULAR PURPOSE AND NONINFRINGEMENT. IN NO EVENT SHALL THE
AUTHORS OR COPYRIGHT HOLDERS BE LIABLE FOR ANY CLAIM, DAMAGES OR OTHER
LIABILITY, WHETHER IN AN ACTION OF CONTRACT, TORT OR OTHERWISE, ARISING FROM,
OUT OF OR IN CONNECTION WITH THE SOFTWARE OR THE USE OR OTHER DEALINGS IN THE
SOFTWARE.
\end{codeInput}

and an even longer code snippet, an SEO React TypeScript component:

\begin{codeInput}{jsx}{Seo.tsx}{A basic SEO React component.}
import * as React from "react"
import { Helmet } from "react-helmet"
import { useStaticQuery, graphql } from "gatsby"
import { siteMetadata } from "../../gatsby-config"

export interface ISeoProps {
  title: string
  description?: string
}

function Seo(props: ISeoProps) {
  const { description, title } = props
  const { site } = useStaticQuery(
    graphql`
      query {
        site {
          siteMetadata {
            title
            description
            author
          }
        }
      }
    `
  )

  return (
    <Helmet>
      {/* General tags */}
      <title>{title}</title>
      <meta
        name="description"
        content={description || siteMetadata.description || ""}
      />
      <meta property="og:title" content={title} />
      <meta
        property="og:description"
        content={description || siteMetadata.description || ""}
      />

      {/* Twitter Card tags */}
      <meta name="twitter:card" content="summary_large_image" />
      <meta name="twitter:creator" content={site.siteMetadata?.author || ""} />
      <meta name="twitter:title" content={title} />
      <meta
        name="twitter:description"
        content={description || siteMetadata.description || ""}
      />
    </Helmet>
  )
}

export default Seo
\end{codeInput}

\begin{codeInput}{sass}{toasts.scss}{A small SASS file.}
.Toastify__progress-bar--default {
  background: \$primary !important;
}
\end{codeInput}

You can also include code from a file. For example, here's a SCSS file:

\codeFromFile{sass}{\_toasts.scss}{The custom styling for the app's toasts.}{./snippets/_toasts.scss}

\codeFromFile{jsx}{Seo.tsx}{The custom styling for the app's toasts.}{./snippets/Seo.tsx}

\minisec{Figures}

Figures. Figures are also important. Here is an image:

\standardfigure{\textwidth/2}{folder-one/example}{Example caption for the image.}

They'll show up automatically in the list of figures.

\minisec{Emojis}

Emojis can also be fun to include in a book. You can use them by the custom commands that are included in this .tex file. Here are some examples:

Soup: \soup
Beers: \beers
Party Popper: \partypopper

That's about it! Those should be all the components you need to write an amazing software engineering book. Good luck!

The full code is uploaded on \cite{JoshiGit2025}.
Also the videos of selected topic availabel on \cite{JoshiYouTube2025}.
 
As discussed by Joshi (2025) \cite{JoshiIJFMR2025}, the integration of Generative AI with Big Data offers significant improvements in financial risk management.

For more read the papers: \cite{JoshiCSEIT2025}, \cite{JoshiIJFMR2025}, \cite{JoshiIJFMR2025}, \cite{JoshiIJIREM2025}

\section*{References}
\begin{thebibliography}{9}

	
	\bibitem{JoshiGit2025}
	Satyadhar Joshi, 
	\textit{Satyadharjoshi GIT Repository}, 
	Accessed: Jan. 19, 2025. [Online]. Available: \url{https://github.com/satyadharjoshi}
	
	\bibitem{JoshiIJFMR2025}
	Satyadhar Joshi, 
	\textit{The Synergy of Generative AI and Big Data for Financial Risk: Review of Recent Developments}, 
	IJFMR - International Journal For Multidisciplinary Research, Volume 7, Issue 1, Accessed: Jan. 19, 2025. [Online]. Available: \url{https://www.ijfmr.com/research-paper.php?id=35488}
	
	\bibitem{JoshiIJIREM2025}
	Satyadhar Joshi, 
	\textit{Implementing Gen AI for Increasing Robustness of US Financial and Regulatory System}, 
	International Journal of Innovative Research in Engineering and Management, Volume 11, Issue 6, pp. 175–179, Jan. 2025. doi: 10.55524/ijirem.2024.11.6.19.
	
	\bibitem{JoshiCSEIT2025}
	Satyadhar Joshi, 
	\textit{Review of Gen AI Models for Financial Risk Management}, 
	International Journal of Scientific Research in Computer Science, Engineering and Information Technology, Volume 11, Issue 1, pp. 709–723, Jan. 2025. doi: 10.32628/CSEIT2511114.
	
	\bibitem{JoshiYouTube2025}
	Satyadhar Joshi, 
	\textit{Shivbhaktajoshi YouTube Channel}, 
	Accessed: Jan. 19, 2025. [Online]. Available: \url{https://www.youtube.com/user/shivbhaktajoshi}
	
\end{thebibliography}





\chapter{Data Engineering and Data lakes for Gen AI  }
In a software book, it's often nice to list chapter objectives at the start of the chapter. I do it this way:
\section{How Data is stored}
hi
\minisec{Learn Big Data and Vector Data Bases}
\begin{arrows}
	\item Some objective
	\item Some other objective
	\item Another objective
\end{arrows}

\minisec{Code Snippets}

Code snippets are of course also essential in a dev book. Here is a code snippet:

\begin{codeInput}{bash}{terminal}{Installing Express via npm.}
	npm install express
\end{codeInput}


\chapter{Agentic Design of Full stack Gen AI system for fin risk analytics}
In a software book, it's often nice to list chapter objectives at the start of the chapter. I do it this way:
\section{Intro to Full Stack Python Paradigm}

\section{Agent-Oriented Architectures for Financial Data Pipelines}
hi
\minisec{Learn Python for Design}
\begin{arrows}
	\item Some objective
	\item Some other objective
	\item Another objective
\end{arrows}

\minisec{Code Snippets}

Code snippets are of course also essential in a dev book. Here is a code snippet:

\begin{codeInput}{bash}{terminal}{Installing Express via npm.}
	npm install express
\end{codeInput}




\chapter{ Using Public Facing LLM Models Like ChatGPTs }
In a software book, it's often nice to list chapter objectives at the start of the chapter. I do it this way:
\section{ChatGPT 101}
\minisec{Chapter Objectives}
\begin{arrows}
	\item Some objective
	\item Some other objective
	\item Another objective
\end{arrows}

\minisec{Code Snippets}

Code snippets are of course also essential in a dev book. Here is a code snippet:

\begin{codeInput}{bash}{terminal}{Installing Express via npm.}
	npm install express
\end{codeInput}




\chapter{Gen AI in Credit Risk (same paper logistic Regression) Huggin face}
In a software book, it's often nice to list chapter objectives at the start of the chapter. I do it this way:
\section{Intro to Credit Risk}
\minisec{Chapter Objectives}
\begin{arrows}
	\item Some objective
	\item Some other objective
	\item Another objective
\end{arrows}

\minisec{Code Snippets}

Code snippets are of course also essential in a dev book. Here is a code snippet:

\begin{codeInput}{bash}{terminal}{Installing Express via npm.}
	npm install express
\end{codeInput}




\chapter{Gen AI in Market Risk (gan vae part 1, then calculate VaR)}
In a software book, it's often nice to list chapter objectives at the start of the chapter. I do it this way:

\section{Intro to Market Risk}
hi
\minisec{Chapter Objectives}
\begin{arrows}
	\item Some objective
	\item Some other objective
	\item Another objective
\end{arrows}

\minisec{Code Snippets}

Code snippets are of course also essential in a dev book. Here is a code snippet:

\begin{codeInput}{bash}{terminal}{Installing Express via npm.}
	npm install express
\end{codeInput}

\section{Models in Market Risk }
hi

\chapter{Gen AI for Interest Rate modeling (gan vae part 2)}
In a software book, it's often nice to list chapter objectives at the start of the chapter. I do it this way:
\section{Vasisec Model}
hi

\minisec{Chapter Objectives}
\begin{arrows}
	\item Some objective
	\item Some other objective
	\item Another objective
\end{arrows}

\minisec{Code Snippets}

Code snippets are of course also essential in a dev book. Here is a code snippet:

\begin{codeInput}{bash}{terminal}{Installing Express via npm.}
	npm install express
\end{codeInput}




\chapter{Gen AI in Structured Finance (to write- you have the subtopics already - treasury MBS etc models)}
In a software book, it's often nice to list chapter objectives at the start of the chapter. I do it this way:
\section{Intro to MBS CDS}
\minisec{Chapter Objectives}
\begin{arrows}
	\item Some objective
	\item Some other objective
	\item Another objective
\end{arrows}

\minisec{Code Snippets}

Code snippets are of course also essential in a dev book. Here is a code snippet:

\begin{codeInput}{bash}{terminal}{Installing Express via npm.}
	npm install express
\end{codeInput}




\chapter{Gen AL Prompt Engineering for Financial Risk (use the one you have)}
In a software book, it's often nice to list chapter objectives at the start of the chapter. I do it this way:
\section{Efficiency and Saving money}
\minisec{Chapter Objectives}
\begin{arrows}
	\item Some objective
	\item Some other objective
	\item Another objective
\end{arrows}

\minisec{Code Snippets}

Code snippets are of course also essential in a dev book. Here is a code snippet:

\begin{codeInput}{bash}{terminal}{Installing Express via npm.}
	npm install express
\end{codeInput}




\chapter{Gen AI in Model Implementation for Financial Risk, (todo)} \label{chap:genai}

2. Add reference to the chapter at the end

At the end of the chapter, you can add a reference to it like this:

\section*{Conclusion}
In this chapter, we discussed the implementation of Generative AI models for financial risk prediction. For further details on the methodologies used, please refer to Chapter \ref{chap:genai}.

3. Using \nameref for a textual reference

Alternatively, if you want to reference the chapter by its name, you can use \nameref:

\section*{Conclusion}
As we have seen in the chapter titled \nameref{chap:genai}, the integration of Generative AI plays a crucial role in enhancing financial risk prediction models.





As discussed by Joshi (2025) \cite{Joshi2025}, the integration of Generative AI with Big Data offers significant improvements in financial risk management.

\section*{References}
\begin{thebibliography}{9}
	\bibitem{Joshi2025}
	Satyadhar Joshi,
	\textit{The Synergy of Generative AI and Big Data for Financial Risk: Review of Recent Developments},
	International Journal For Multidisciplinary Research, 2025, Volume 7, Issue 1, pp. 1-10.
\end{thebibliography}




\chapter*{Afterword}
\addcontentsline{toc}{chapter}{Afterword}

\section*{You've Done It!}

Something, something.

Cheers! \beers

-Jane

\chapter*{Credits and Thanks}
\addcontentsline{toc}{chapter}{Credits}

Credit where credit is due! (Note that I am not sponsored or supported by any of these platforms or individuals in anyway):

\begin{enumerate}
\item Netlify, for their awesome "feels like stealing" free tier
\item Bitbucket, for their great UI and tooling, including Bitbucket Pipelines
\item Digital Ocean, for the sheer ease of to start up a Linux instance with a few clicks
\item \link{Dabolus on DeviantArt}{https://www.deviantart.com/dabolus}, for all of those juicy hi-res emoji PNGs that I've used generously throughout the book!
\end{enumerate}

% appendices
\begin{appendices}

\chapter{Appendix 1}

Here is appendix 1.

\chapter{Appendix 2}

Here is appendix 2.

\end{appendices}

% index!
\printindex
\addcontentsline{toc}{chapter}{Index}


% about the author
About the Author

\standardfigure{\textwidth/2}{about/author}{Jane Doe}

Jane Doe is a Senior Full Stack Developer with over 10 years of programming experience, the last 7 of which were in industry. When she's not writing code, building SaaS Products, or teaching full stack software engineering, she can be found painting, writing flute music. She is originally from New York City, but currently resides in a country, in a town, on planet earth.

\end{document}
